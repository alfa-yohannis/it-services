\documentclass[aspectratio=169, table]{beamer}

\usepackage[utf8]{inputenc}
\usepackage{listings} 

\usetheme{Pradita}

\subtitle{MTI104 - IT Services}

\title{Session-01:\\\LARGE{ITIL’s Management of Practices \\}}
\date[Serial]{\scriptsize {PRU/SPMI/FR-BM-18/0222}}
\author[Pradita]{\small{\textbf{Alfa Yohannis}}}

\begin{document}

\frame{\titlepage}

\begin{frame}
	\frametitle{Introduction to ITIL 4 Practices}
	
	\begin{itemize}
		\item ITIL 4 introduces practices, replacing processes and functions.
		\item Previous concepts were often complex and had a steep learning curve.
		\item Processes represented activities needed to turn input into output.
		\item Functions were team structures providing resources for processes.
		\item Practices combine processes and functions into a unified approach.
	\end{itemize}
	
\end{frame}

% Slide 2
\begin{frame}
	\frametitle{Definition of a Practice in ITIL 4}
	
	\begin{itemize}
		\item A practice is a set of organizational resources for performing work or achieving an objective.
		\item It includes people, infrastructure, software, and processes.
		\item Resources are aligned to achieve a specific objective.
		\item Practices are not general; they are designed for specific purposes.
		\item Example: A practice for making burgers is not suitable for making burritos.
	\end{itemize}
	
\end{frame}

% Slide 3
\begin{frame}
	\frametitle{Categories of Practices in ITIL 4}
	
	\begin{itemize}
		\item Practices are integral to the service value system.
		\item Each practice supports multiple value chain activities.
		\item Practices include resources based on the four dimensions of service management.
		\item ITIL practices are categorized into three parts:
		\begin{itemize}
			\item General management practices
			\item Service management practices
			\item Technical management practices
		\end{itemize}
	\end{itemize}
	
\end{frame}

% Slide 4
\begin{frame}
	\frametitle{General Management Practices}
	
	\begin{itemize}
		\item General management practices are generic and common across frameworks.
		\item These practices are adapted from general business management domains.
		\item There are fourteen general management practices in ITIL:
		\begin{itemize}
			\item Architecture management
			\item Continual improvement
			\item Information security management
		\end{itemize}
	\end{itemize}
	
\end{frame}

% Slide 5
\begin{frame}
	\frametitle{General Management Practices (Continued)}
	
	\begin{itemize}
		\item Additional general management practices include:
		\begin{itemize}
			\item Knowledge management
			\item Measurement and reporting
			\item Organizational change management
			\item Portfolio management
			\item Project management
		\end{itemize}
		\item In the ITIL Foundation exam, specific practices are in scope:
		\begin{itemize}
			\item Continual improvement
			\item Information security management
		\end{itemize}
	\end{itemize}
	
\end{frame}

% Slide 6
\begin{frame}
	\frametitle{Service Management Practices}
	
	\begin{itemize}
		\item Service management practices evolved in ITSM organizations like IBM and HP.
		\item These practices were a collection of best practices in the industry.
		\item Service management practices in ITIL 4:
		\begin{itemize}
			\item Availability management
			\item Business analysis
			\item Capacity and performance management
		\end{itemize}
		\item These practices have become the de facto standard for service management.
	\end{itemize}
	
\end{frame}

% Slide 7
\begin{frame}
	\frametitle{Service Management Practices (Continued)}
	
	\begin{itemize}
		\item Additional service management practices include:
		\begin{itemize}
			\item Change control
			\item Incident management
			\item IT asset management
			\item Monitoring and event management
			\item Problem management
		\end{itemize}
		\item In the ITIL Foundation exam, specific practices are in scope:
		\begin{itemize}
			\item Change control
			\item Incident management
		\end{itemize}
	\end{itemize}
	
\end{frame}

% Slide 8
\begin{frame}
	\frametitle{Technical Management Practices}
	
	\begin{itemize}
		\item Technical management practices originated in technology-based organizations.
		\item They provide a linkage between technical activities and workflow-related activities.
		\item Technical management practices have been adapted for service management.
		\item Three technical management practices in ITIL 4:
		\begin{itemize}
			\item Deployment management
			\item Infrastructure and platform management
			\item Software development and management
		\end{itemize}
		\item Deployment management is in scope for the ITIL Foundation exam.
	\end{itemize}
	
\end{frame}

\begin{frame}
	\frametitle{Multiple Choice Question}
	
	\textbf{Which of the following practices does not figure in the technical management practices?}
	
	\begin{enumerate}[A.]
		\item Deployment management
		\item Release and deployment management
		\item Infrastructure and platform management
		\item Software development and management
	\end{enumerate}
	
\end{frame}


\end{document}
