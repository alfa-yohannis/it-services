\documentclass[aspectratio=169, table]{beamer}

\usepackage[utf8]{inputenc}
\usepackage{listings} 

\usetheme{Pradita}

\subtitle{MTI104 - IT Services}

\title{Session-01:\\\LARGE{Practices to Manage Stakeholders\\}}
\date[Serial]{\scriptsize {PRU/SPMI/FR-BM-18/0222}}
\author[Pradita]{\small{\textbf{Alfa Yohannis}}}

\begin{document}

\frame{\titlepage}

\begin{frame}
	\frametitle{Introduction to ITIL Stakeholder Management}
	
	\begin{itemize}
		\item Services require continuous support and external parties.
		\item External parties include customers, users, sponsors, and suppliers.
		\item Managing work activities internally reduces complexity.
		\item Handling stakeholders involves managing relationships.
		\item Practices to manage stakeholders include:
		\begin{itemize}
			\item Relationship management
			\item Supplier management
			\item Service level management
		\end{itemize}
	\end{itemize}
	
\end{frame}

\begin{frame}
	\frametitle{Relationship Management Overview}
	
	\begin{itemize}
		\item Not all stakeholders are equally important.
		\item Stakeholders are engaged at different levels:
		\begin{itemize}
			\item Strategic
			\item Tactical
			\item Operational
		\end{itemize}
		\item Example: IT company with MS Teams collaboration.
		\item Strategic relationships focus on long-term collaboration.
		\item Tactical relationships involve medium-term agreements.
		\item Operational transactions are case-by-case with minimal relationship.
	\end{itemize}
	
\end{frame}

\begin{frame}
	\frametitle{Relationship Management in ITIL}
	
	\begin{itemize}
		\item Relationship management deals with strategic and tactical levels.
		\item Purpose: Establish and nurture links with stakeholders.
		\item Ensures relationships are fruitful for all involved.
		\item Follows a lifecycle through stages:
		\begin{itemize}
			\item Identification
			\item Analysis
			\item Monitoring
			\item Continual Improvement
		\end{itemize}
	\end{itemize}
	
\end{frame}

\begin{frame}
	\frametitle{Value Co-Creation in Relationship Management}
	
	\begin{itemize}
		\item Value is co-created between service provider and customer.
		\item Emphasis on harmony and reducing conflicts.
		\item Key purposes of relationship management:
		\begin{itemize}
			\item Understanding stakeholder requirements.
			\item Prioritization of requirements.
			\item Regular conversations to ensure transparency.
			\item Managing customer satisfaction and complaints.
		\end{itemize}
		\item Value co-creation must be the objective at strategic and tactical levels.
	\end{itemize}
	
\end{frame}

\begin{frame}
	\frametitle{Engagement in Service Value Chain}
	
	\begin{itemize}
		\item Relationship management engages with service value chain activities:
		\item High involvement in planning and design and transition stages.
		\item Medium involvement in obtain/build and improve stages.
		\item Ensures alignment between stakeholder requirements and service value.
		\item Facilitates feedback from customers to improve services.
		\item Builds strong relationships with stakeholders to enhance value delivery.
	\end{itemize}
	
\end{frame}

\begin{frame}
	\frametitle{Supplier Management Overview}
	
	\begin{itemize}
		\item Supplier management is a mature ITIL practice.
		\item A service provider’s service provider is referred to as a supplier.
		\item Suppliers provide essential goods and services.
		\item Purpose: Ensure suppliers and their performance are managed appropriately.
		\item Strategy drawn from customer requirements and service provider capabilities.
		\item Supplier management activities include:
		\begin{itemize}
			\item Evaluation and selection
			\item Contract negotiations
		\end{itemize}
	\end{itemize}
	
\end{frame}

\begin{frame}
	\frametitle{Supplier Management Activities}
	
	\begin{itemize}
		\item Supplier categorization into strategic, tactical, and operational.
		\item Managing performance through regular checks.
		\item KPIs and SLAs are defined during contract negotiations.
		\item Sourcing strategies: Insourcing, outsourcing, single sourcing, and multisourcing.
		\item Example: Application development split among multiple service providers.
		\item Supplier management engages in all stages of the service value chain.
		\item Performance management and improvement are ongoing activities.
	\end{itemize}
	
\end{frame}

\begin{frame}
	\frametitle{Introduction}
	\begin{itemize}
		\item Relationship Management: Builds strategic and tactical relationships.
		\item Supplier Management: Focuses on supplier relationships.
		\item Service Level Management: Engages operationally with stakeholders.
		\item Purpose: Set, monitor, and manage service level targets.
		\item Ensures services meet performance and efficiency targets.
		\item Metrics define expected service quality.
		\item Example: Pizza delivery within 30 minutes.
	\end{itemize}
\end{frame}

\begin{frame}
	\frametitle{Primary Activities}
	\begin{itemize}
		\item Agree on binding service levels (e.g., availability, resolution timelines).
		\item Monitor service levels and ensure they are met.
		\item Regular service reviews to identify root causes of inadequacies.
		\item Log issues and shortcomings to customers and stakeholders.
		\item Service levels measure end-to-end services, not individual components.
	\end{itemize}
\end{frame}

\begin{frame}
	\frametitle{Service Level Agreements (SLAs)}
	\begin{itemize}
		\item SLA: Documented agreement between provider and customer.
		\item Defines services required and expected service levels.
		\item SLAs are appended to contract documents.
		\item Service Level Requirements (SLRs): Customer expectations aligned to business objectives.
		\item Balancing service levels with service costs.
	\end{itemize}
\end{frame}

\begin{frame}
	\frametitle{SLA Details}
	\begin{itemize}
		\item Reference service catalog for SLA definition.
		\item Define multiple facets of service levels.
		\item Ensure SLA aligns with business processes.
		\item SLA document should be simple, clear, and unambiguous.
	\end{itemize}
\end{frame}

\begin{frame}
	\frametitle{The Watermelon SLA Effect}
	\begin{itemize}
		\item Met SLA but failed in critical periods (e.g., month-end processing).
		\item Green on the outside, red on the inside metaphor.
		\item SLA may show high availability but fail when it matters.
		\item Avoid watermelon effect by considering business parameters.
	\end{itemize}
\end{frame}

\begin{frame}
	\frametitle{Service Level Management in SVC}
	\begin{itemize}
		\item Plan: Influences service planning and portfolios.
		\item Design and Transition: Affects future designs based on feedback.
		\item Obtain/Build: Receives service levels and reporting needs.
		\item Engage: Uses service levels for stakeholder engagement and reviews.
		\item Deliver and Support: Needs to maintain agreed service levels.
		\item Improve: Feedback drives improvement activities.
	\end{itemize}
\end{frame}

\begin{frame}
	\frametitle{Question}
	
	Which activity in the service value chain is responsible primarily for providing feedback from customers?
	
	\begin{itemize}
		\item[A.] Deliver and Support
		\item[B.] Obtain/Build
		\item[C.] Engage
		\item[D.] Plan
	\end{itemize}
	
\end{frame}


\end{document}
