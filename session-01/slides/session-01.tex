\documentclass[aspectratio=169, table]{beamer}

\usepackage[utf8]{inputenc}
\usepackage{listings} 

\usetheme{Pradita}

\subtitle{MTI104 - IT Services}

\title{Session-01:\\\LARGE{ITIL Introduction\\and DevOps Overview}}
\date[Serial]{\scriptsize {PRU/SPMI/FR-BM-18/0222}}
\author[Pradita]{\small{\textbf{Alfa Yohannis}}}

\begin{document}

\frame{\titlepage}

\begin{frame}{Introduction to ITIL 4.5}
	\begin{itemize}
		\item ITIL has evolved through multiple versions.
		\item ITIL 4 offers a fresh perspective on service management.
		\item ITIL 4 addresses the blurred line between development and operations.
		\item The framework adapts to the digital age.
		\item ITIL 4 replaces outdated practices from ITIL V3.
		\item It focuses on agility and innovation.
		\item ITIL 4 was developed to remain relevant in the fast-changing IT world.
	\end{itemize}
\end{frame}

\begin{frame}{Why ITIL 4?}
	\begin{itemize}
		\item ITIL's evolution from V3 to 4 reflects industry changes.
		\item Initial concerns were about certification relevance.
		\item ITIL V3 was widely successful but became outdated.
		\item New ITIL 4 framework needed to address digital transformation.
		\item ITIL 4 aims to modernize service management practices.
		\item The refresh was seen as overdue but necessary.
		\item ITIL 4 adapts to new industry dynamics and practices.
	\end{itemize}
\end{frame}

\begin{frame}{ITIL V3 Limitations}
	\begin{itemize}
		\item ITIL V3 was seen as outdated in the digital age.
		\item Its traditional framework lacked agility and dynamism.
		\item It was criticized for its process-driven approach.
		\item The service lifecycle model became irrelevant for modern needs.
		\item ITIL V3's phases were too rigid for fast-paced environments.
		\item The framework was compared to outdated technologies like Nokia's phones.
		\item The need for a more flexible and responsive approach led to ITIL 4.
	\end{itemize}
\end{frame}

\begin{frame}{DevOps and ITIL 4}
	\begin{itemize}
		\item DevOps bridged development and operations.
		\item It replaced traditional blame games with collaboration.
		\item DevOps methodology emphasizes agility and rapid delivery.
		\item ITIL V3's rigid processes conflicted with DevOps practices.
		\item ITIL 4 integrates with DevOps by adopting agile principles.
		\item New frameworks needed to support continuous development and operations.
		\item ITIL 4 adapts to modern workflows and product management needs.
	\end{itemize}
\end{frame}

\begin{frame}{ITIL 4 Key Changes}
	\begin{itemize}
		\item The service lifecycle concept is replaced by the Service Value System.
		\item Introduction of practices that encompass processes and tools.
		\item Functions from ITIL V3 are integrated into practices.
		\item Service definition now includes value co-creation with customers.
		\item Governance has a more explicit role in ITIL 4.
		\item Automation is emphasized for efficiency and integration.
		\item The certification path is updated to reflect new roles and practices.
	\end{itemize}
\end{frame}

\begin{frame}{ITIL 4 Certification Hierarchy}
	\begin{itemize}
		\item ITIL Foundation is the entry-level certification.
		\item ITIL Managing Professional (MP) focuses on service management.
		\item ITIL Strategic Leader (SL) looks at business strategy.
		\item ITIL Master certification is for advanced practitioners.
		\item Certifications cover various aspects of ITIL and digital strategy.
		\item ITIL 4 certifications align with modern service management needs.
		\item The hierarchy helps professionals choose relevant certifications.
	\end{itemize}
\end{frame}

\begin{frame}{ITIL Conclusion}
	\begin{itemize}
		\item ITIL 4 represents a significant update from V3.
		\item It addresses the needs of modern IT environments.
		\item Emphasizes agility, collaboration, and value co-creation.
		\item Adapts to industry shifts towards DevOps and digital transformation.
		\item Provides a more flexible and relevant framework for service management.
		\item Updated certification paths reflect evolving professional roles.
		\item ITIL 4 aims to stay relevant in the fast-evolving IT landscape.
	\end{itemize}
\end{frame}

\begin{frame}
	\frametitle{Brief Overview of DevOps}
	\begin{itemize}
		\item New methodologies often emerge to solve problems.
		\item DevOps was created to address the need for fast solution turnarounds.
		\item Businesses faced issues with incomplete information during development.
		\item DevOps evolved to enhance agility and productivity.
		\item It aims to improve software quality over time.
		\item Operations also benefit from DevOps through increased automation.
		\item Automation transforms mundane tasks into innovative processes.
	\end{itemize}
\end{frame}

\begin{frame}
	\frametitle{The Evolution of DevOps}
	\begin{itemize}
		\item Development was managed through waterfall project management.
		\item ITIL was dominant in operations before DevOps.
		\item DevOps merges development and operations.
		\item Waterfall methodology gave way to Agile practices.
		\item Concerns about ITIL’s role in DevOps were unfounded.
		\item DevOps is not the end for ITIL but a complementary approach.
	\end{itemize}
\end{frame}

\begin{frame}
	\frametitle{What Exactly Is DevOps?}
	\begin{itemize}
		\item Multiple definitions exist for DevOps.
		\item Commonly understood as integrating development and operations teams.
		\item Initially thought to be just about automation or startups.
		\item Increasingly recognized as a cultural change.
		\item DevOps involves more than just combining teams.
	\end{itemize}
\end{frame}

\begin{frame}
	\frametitle{DevOps Explained with an Example}
	\begin{itemize}
		\item Example: Rollback of a failed deployment.
		\item Root cause analysis and fixing involve multiple stages.
		\item Blame culture vs. DevOps blameless culture.
		\item In DevOps, responsibility is shared among the team.
		\item Mistakes are viewed as part of human nature.
	\end{itemize}
\end{frame}

\begin{frame}
	\frametitle{Why DevOps?}
	\begin{itemize}
		\item Evolution from sequential waterfall to Agile methodologies.
		\item Agile introduced flexibility but still had limitations.
		\item DevOps enhances Agile with automation for faster delivery.
		\item Automation improves efficiency beyond what Agile alone can offer.
	\end{itemize}
\end{frame}

\begin{frame}
	\frametitle{Benefits of Transforming into DevOps}
	\begin{itemize}
		\item Enhanced working culture and technology.
		\item Organizations like Amazon set new benchmarks in deployment.
		\item Amazon statistics: 1,079 deployments/hour, 75% reduction in outages.
		\item Automation and reduced downtime are key benefits.
	\end{itemize}
\end{frame}

\begin{frame}
	\frametitle{DevOps Principles}
	\begin{itemize}
		\item Principles are evolving and include CALMS.
		\item CALMS stands for:
		\begin{itemize}
			\item Culture
			\item Automation
			\item Lean
			\item Measurement
			\item Sharing
		\end{itemize}
	\end{itemize}
\end{frame}

\begin{frame}
	\frametitle{DevOps Scope and Elements}
	\begin{itemize}
		\item DevOps is a cultural change, not a framework.
		\item It includes people, process, and technology.
		\item Implementing DevOps spans the entire software lifecycle.
		\item Key elements: People, Process, Technology.
		\item DevOps practices can be applied to various parts of the software industry.
	\end{itemize}
\end{frame}

\begin{frame}
	\frametitle{Brief Overview of DevOps}
	\begin{itemize}
		\item Conjunction of development and operations
		\item Change in software delivery culture
		\item People at the heart of transformation
		\item Integration of development and operations teams
		\item Resolving conflicts and mistrust
		\item Balancing development and operations priorities
		\item Creating communication channels
	\end{itemize}
\end{frame}

\begin{frame}
	\frametitle{Integration of Development and Operations}
	\begin{itemize}
		\item Teams amalgamated for cultural change
		\item Development integrates with operations
		\item Reduced issues in CAB
		\item Operational familiarity with development
		\item Better software maintainability
		\item Progressive collaboration
		\item Increased trust and efficiency
	\end{itemize}
\end{frame}

\begin{frame}
	\frametitle{Conflict between Development \& Operations}
	\begin{itemize}
		\item Development team vs. operations team
		\item Cliff-like separation between teams
		\item Unpredictable activities in between
		\item Miscommunication and mistrust
		\item Need for a bridge between teams
		\item Creation of shared responsibility
	\end{itemize}
\end{frame}

\begin{frame}
	\frametitle{Balancing Priorities}
	\begin{itemize}
		\item Development focus: new features
		\item Operations focus: stability
		\item Development introduces change
		\item Operations ensure environment stability
		\item Continuous changes impact stability
		\item Balance between innovation and stability
		\item Continuous improvements and maintenance
	\end{itemize}
\end{frame}

\begin{frame}
	\frametitle{Creating Channels of Communication}
	\begin{itemize}
		\item Merging development and operations
		\item Shared responsibility for smooth operations
		\item Communication within team members
		\item Joint work on development and deployment
		\item Reduction of mistrust
		\item Improved deployment confidence
		\item Enhanced collaboration
	\end{itemize}
\end{frame}

\begin{frame}
	\frametitle{Process}
	\begin{itemize}
		\item Key to project success
		\item Focus on processes over automation
		\item Automation should follow processes
		\item Define processes first
		\item Align with DevOps architecture
		\item Tools and automation support processes
		\item Avoid process-driven tool adoption
	\end{itemize}
\end{frame}

\begin{frame}
	\frametitle{Adapting Processes for DevOps}
	\begin{itemize}
		\item Adjust processes for new objectives
		\item Transition from traditional methodologies
		\item Agile methodologies preferred
		\item Flexibility and adaptation in project management
		\item Importance of process in DevOps
		\item Agile principles and practices
		\item Rejigging processes for efficiency
	\end{itemize}
\end{frame}

\begin{frame}
	\frametitle{Technology}
	\begin{itemize}
		\item Technology as a key DevOps element
		\item Importance of automation
		\item People and processes must come first
		\item Technology supports efficiency
		\item Proper synchrony of roles and processes
		\item Tools enhance DevOps practices
		\item Avoid technology-driven development
	\end{itemize}
\end{frame}

\begin{frame}
	\frametitle{DevOps Practices}
	\begin{itemize}
		\item Continuous Integration
		\item Continuous Delivery
		\item Infrastructure as Code
		\item Continuous Monitoring
		\item Integration and deployment practices
		\item Automation in software management
		\item Real-time monitoring and feedback
	\end{itemize}
\end{frame}

\begin{frame}
	\frametitle{Continuous Integration}
	\begin{itemize}
		\item Regular code integration
		\item Detect conflicts early
		\item Minimize rework
		\item Frequent integration checks
		\item Automated builds and tests
		\item Resolve conflicts in real-time
		\item Maintain code quality
	\end{itemize}
\end{frame}

\begin{frame}
	\frametitle{Continuous Delivery}
	\begin{itemize}
		\item Automated deployment to production
		\item Maintain deployable state
		\item Reduce deployment effort
		\item Frequent and reliable releases
		\item Rapid feedback and fixes
		\item Minimize deployment risks
		\item Improve feature delivery speed
	\end{itemize}
\end{frame}

\begin{frame}
	\frametitle{Infrastructure as Code}
	\begin{itemize}
		\item Automate infrastructure management
		\item Define infrastructure with code
		\item Version control for infrastructure
		\item Consistency and repeatability
		\item Reduce manual interventions
		\item Manage resources through configuration
		\item Enable scalable infrastructure setup
	\end{itemize}
\end{frame}

\begin{frame}
	\frametitle{Continuous Monitoring}
	\begin{itemize}
		\item Ongoing performance analysis
		\item Real-time metrics collection
		\item Identify issues promptly
		\item Maintain system stability
		\item Proactive issue resolution
		\item Monitor application health
		\item Improve system performance
	\end{itemize}
\end{frame}

\begin{frame}
	\frametitle{DevOps Conclusion}
	\vspace{15pt}
	\begin{itemize}
		\item DevOps integrates development and operations to improve agility.
		\item Evolved from traditional methodologies like Waterfall and ITIL.
		\item Emphasizes cultural change and shared responsibility.
		\item Enhances Agile practices with automation for faster delivery.
		\item Benefits include increased efficiency, reduced downtime, and improved software quality.
		\item Key principles: Culture, Automation, Lean, Measurement, Sharing (CALMS).
		\item Effective DevOps involves people, processes, and technology across the software lifecycle.
	\end{itemize}
\end{frame}


\end{document}
